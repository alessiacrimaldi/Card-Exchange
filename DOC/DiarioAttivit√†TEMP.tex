\documentclass{article}
\usepackage[utf8]{inputenc}

\title{ggg}
\author{davide.fermi }
\date{January 2023}

\begin{document}

    \maketitle

    \section{SPRINT 02}

    \begin{itemize}
        \item Inizio sprint: \textit{15/01/2023}
        \item Fine sprint: \textit{21/01/2023}
    \end{itemize}

    \begin{itemize}
        \item \textbf{Sprint Goal:}
        \begin{indent}
            \newline #TODO
        \end{indent}
    \end{itemize}

    \begin{itemize}
        \item \textbf{RUOLI.}
        \newline \textbf{Product Owner:} Matteo Sacco
        \newline \textbf{Scrum Master:} Davide Fermi
        \newline \textbf{Development team:} Alessia Crimaldi, Alessio Arcara
    \end{itemize}


    \subsection{Sprint Planning}
    \begin{itemize}
        \item \textbf{Data:} 15/01/2023
        \newline \textbf{Durata:} 110 min.
        \newline \textbf{Partecipanti:} Scrum Master (Davide Fermi), Product Owner (Matteo Sacco), Dev 1 (Alessia Crimaldi), Dev 2 (Alessio Arcara)
        \newline \textbf{Sintesi:}
        \begin{itemize}
            \item \textbf{PO:} Illustra e descrive casi d'Uso a Team
            \item \textbf{SM:} Conduce PokerPlan con Team per assegnazione story points
            \item \textbf{Team:} Stima l'effort necessario per ogni UC
            \item \textbf{Team:} Stima l'effort totale disponibile
            \item \textbf{Devs:} Creazione tasks da UC
            \item \textbf{Team:} Stima l'effort necessario per ogni task
        \end{itemize}


        \begin{itemize}
            \item \textbf{BOARD}.
            \newline
            \textbf{Stato dopo lo Sprint Planning:}
            \newline
            \newline
            \begin{tabular}{ | p{3.5cm} | p{1.5cm} | p{5cm} | p{1.6cm} | p{1.6cm} | }
                \hline
                \textbf{UC raffinati}
                & \textbf{Product Backlog}
                & \textbf{Sprint Backlog}
                & \textbf{Sprint In progress}
                & \textbf{Sprint Done} \\
                \hline
                \textbf{UC01:} Come visitatore voglio selezionare un gioco e vedere l'elenco di tutte le sue carte
                & & \textbf{Task1 [UC01]\#2:} Creare il modello per le carte di tutti i giochi & & \\
                \hline
                \textbf{UC02:} Come visitatore voglio poter fare delle ricerche per vedere un elenco di carte filtrato
                & & \textbf{Task2 [UC01]\#3:} Importare dai JSON le carte dei giochi & & \\
                \hline
                \textbf{UC03:} Come visitatore voglio selezionare una carta e vederne i dettagli
                & & \textbf{Task3 [UC01]\#4:} Inizializzare il DB con i JSON caricati & & \\
                \hline
                & & \textbf{Task4 [UC01]\#5:} Creare RemoteService per fetch delle carte di quel gioco & & \\
                \hline
                & & \textbf{Task5 [UC01]\#6:} Creare HomeView per la visualizzazione delle carte & & \\
                \hline
                & & \textbf{Task6 [UC01]\#7:} Creare Composite Card per la visualizzazione carta & & \\
                \hline
                & & \textbf{Task1 [UC02]\#10:} Creare filtri specifici di quel gioco & & \\
                \hline
                & & \textbf{Task2 [UC02]\#11:} Filtrare l'array delle carte in base ai filtri specificati & & \\
                \hline
                & & \textbf{Task3 [UC02]\#12:} Rimuovere dalla pagina HomeView le carte che non corrispondono ai filtri & & \\
                \hline
                & & \textbf{Task1 [UC03]\#14:} Creazione CardPlace & & \\
                \hline
                & & \textbf{Task2 [UC03]\#15:} Connessione RemoteService FetchCard nel caso l'assunzione sia errata & & \\
                \hline
                & & \textbf{Task3 [UC03]\#16:} Creazione CardView e popolamento delle informazioni & & \\
                \hline
                & & \textbf{Task4 [UC03]\#17:} Gestione del cambio di pagina da HomeView a CardView & & \\
                \hline
            \end{tabular}
            \newpage
            \textbf{Stato finale:}
            \newline
            \newline
            \begin{tabular}{ | p{2.4cm} | p{1.4cm} | p{2.7cm} | p{3.4cm} | p{3cm} | }
                \hline
                \textbf{UC raffinati}
                & \textbf{Product Backlog}
                & \textbf{Sprint Backlog}
                & \textbf{Sprint In progress}
                & \textbf{Sprint Done} \\
                \hline
                \textbf{UC01:} Come visitatore voglio selezionare un gioco e vedere l'elenco di tutte le sue carte
                & & \textbf{Task1 [UC03]\#14:} Creazione CardPlace
                & \textbf{Task2 [UC02]\#11:} Filtrare l'array delle carte in base ai filtri specificati
                & \textbf{Task1 [UC01]\#2:} Creare il modello per le carte di tutti i giochi \\
                \hline
                \textbf{UC02:} Come visitatore voglio poter fare delle ricerche per vedere un elenco di carte filtrato
                & & \textbf{Task2 [UC03]\#15:} Connessione RemoteService FetchCard nel caso l'assunzione sia errata
                & \textbf{Task3 [UC02]\#12:} Rimuovere dalla pagina HomeView le carte che non corrispondono ai filtri
                & \textbf{Task4 [UC01]\#5:} Creare RemoteService per fetch delle carte di quel gioco \\
                \hline
                \textbf{UC03:} Come visitatore voglio selezionare una carta e vederne i dettagli
                & & \textbf{Task3 [UC03]\#16:} Creazione CardView e popolamento delle informazioni
                & \textbf{Task6 [UC01]\#7:} Creare Composite Card per la visualizzazione carta
                & \textbf{Task5 [UC01]\#6:} Creare HomeView per la visualizzazione delle carte \\
                \hline
                & & \textbf{Task4 [UC03]\#17:} Gestione del cambio di pagina da HomeView a CardView
                & & \textbf{Task2 [UC01]\#3:} Importare dai JSON le carte dei giochi \\
                \hline
                & & & & \textbf{Task1 [UC02]\#10:} Creare filtri specifici di quel gioco  \\
                \hline
                & & & & \textbf{Task3 [UC01]\#4:} Inizializzare il DB con i JSON caricati \\
                \hline
            \end{tabular}
        \end{itemize}



        \subsection{Daily Scrum}
        \begin{itemize}
            \item \textbf{Data:} 16/01/2023
            \newline \textbf{Durata:} 20 min.
            \newline \textbf{Partecipanti:} Scrum Master (Davide Fermi), Dev 1 (Alessia Crimaldi), Dev 2 (Alessio Arcara)
            \newline \textbf{Percezione attuale dello SG:} Positiva (Dev 1), Positiva (Dev 2)
        \end{itemize}
        \begin{itemize}
            \item \textbf{Data:} 17/01/2023
            \newline \textbf{Durata:} 15 min.
            \newline \textbf{Partecipanti:} Scrum Master (Davide Fermi), Dev 1 (Alessia Crimaldi), Dev 2 (Alessio Arcara)
            \newline \textbf{Percezione attuale dello SG:} Positiva (Dev 1), Positiva (Dev 2)
        \end{itemize}
        \begin{itemize}
            \item \textbf{Data:} 18/01/2023
            \newline \textbf{Durata:} 15 min.
            \newline \textbf{Partecipanti:} Scrum Master (Davide Fermi), Dev 1 (Alessia Crimaldi), Dev 2 (Alessio Arcara)
            \newline \textbf{Percezione attuale dello SG:} Positiva (Dev 1), Positiva (Dev 2)
        \end{itemize}
        \begin{itemize}
            \item \textbf{Data:} 19/01/2023
            \newline \textbf{Durata:} 15 min.
            \newline \textbf{Partecipanti:} Scrum Master (Davide Fermi), Dev 1 (Alessia Crimaldi), Dev 2 (Alessio Arcara)
            \newline \textbf{Percezione attuale dello SG:} Positiva (Dev 1), Positiva (Dev 2)
        \end{itemize}
        \begin{itemize}
            \item \textbf{Data:} 20/01/2023
            \newline \textbf{Durata:} 15 min.
            \newline \textbf{Partecipanti:} Scrum Master (Davide Fermi), Dev 1 (Alessia Crimaldi), Dev 2 (Alessio Arcara)
            \newline \textbf{Percezione attuale dello SG:} Positiva (Dev 1), Positiva (Dev 2)
        \end{itemize}
        \begin{itemize}
            \item \textbf{Data:} 21/01/2023
            \newline \textbf{Durata:} 15 min.
            \newline \textbf{Partecipanti:} Scrum Master (Davide Fermi), Dev 1 (Alessia Crimaldi), Dev 2 (Alessio Arcara)
            \newline \textbf{Percezione attuale dello SG:} Positiva (Dev 1), Positiva (Dev 2)
        \end{itemize}


    \end{document}
