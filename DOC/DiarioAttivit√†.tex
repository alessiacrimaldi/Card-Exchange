\documentclass{article}
\usepackage[utf8]{inputenc}

\title{Diario del progetto}
% \author{Alessia Crimaldi}
\date{January 2023}

\begin{document}

    \maketitle

    \section{SPRINT 01}

    \begin{itemize}
        \item Inizio sprint: \textit{08/01/2023}
        \item Fine sprint: \textit{14/01/2023}
    \end{itemize}

    \begin{itemize}
        \item \textbf{Sprint Goal:}
        \begin{indent}
            \newline Permettere ad un visitatore di effettuare tutte le funzionalita' base dell'app, ossia di avere una corretta ed efficace importazione e gestione del catalogo carte con le relative ricerche e dettagli.
        \end{indent}
    \end{itemize}

    \begin{itemize}
        \item \textbf{RUOLI.}
        \newline \textbf{Product Owner:} Davide Fermi
        \newline \textbf{Scrum Master:} Alessia Crimaldi
        \newline \textbf{Development team:} Alessio Arcara, Matteo Sacco
    \end{itemize}

    \begin{itemize}
        \item \textbf{BOARD}.
        \newline
        \textbf{Stato iniziale:}
        \newline
        \newline
        \begin{tabular}{ |c|c|c|c|c| }
            \hline
            UC raffinati      &   Product Backlog                         &   UC Effort   &   Sprint Backlog   &   Task Effort \\
            \hline
                              &   \textbf{UC01:} Come visitatore voglio   &               &                    &               \\
            \tababularnewline &   selezionare un gioco e vedere           &               &                    &               \\
            \tababularnewline &   l'elenco di tutte le sue carte          &               &                    &               \\
            \hline
                              &   \textbf{UC02:} Come visitatore voglio   &               &                    &               \\
            \tababularnewline &   poter fare delle ricerche per           &               &                    &               \\
            \tababularnewline &   vedere un elenco di carte filtrato      &               &                    &               \\
            \hline
                              &   \textbf{UC03:} Come visitatore voglio   &               &                    &               \\
            \tababularnewline &   selezionare una carta e                 &               &                    &               \\
            \tababularnewline &   vederne i dettagli                      &               &                    &               \\
            \hline
        \end{tabular}
        \newpage
        \textbf{Stato dopo lo sprint planning}
        \newline
        \newline
        \begin{tabular}{ |c|c|c|c|c| }
            \hline
            UC raffinati   &   Product Backlog   &   UC Effort   &   Sprint Backlog   &   Task Effort \\
            \hline
            UC01           &                     &               &   Task 1 per UC1   &               \\
            \hline
                           &                     &               &   Task 2 per UC1   &               \\
            \hline
                           &                     &               &   Task 3 per UC1   &               \\
            \hline
            UC02           &                     &               &   Task 1 per UC2   &               \\
            \hline
                           &                     &               &   Task 2 per UC2   &               \\
            \hline
                           &                     &               &   Task 3 per UC2   &               \\
            \hline
            UC03           &                     &               &   Task 1 per UC3   &               \\
            \hline
                           &                     &               &   Task 2 per UC3   &               \\
            \hline
                           &                     &               &   Task 3 per UC3   &               \\
            \hline
        \end{tabular}
    \end{itemize}

    \subsection{Daily Scrum}
    \begin{itemize}
        \item \textbf{Data:} 09/01/2023
        \newline \textbf{Durata:} 15 min.
        \newline \textbf{Partecipanti:} Scrum Master (Alessia Crimaldi), Dev 1 (Alessio Arcara)
        \newline \textbf{Percezione attuale dello SG:} Positiva (Dev 1)
    \end{itemize}
    \begin{itemize}
        \item \textbf{Data:} 10/01/2023
        \newline \textbf{Durata:} 15 min.
        \newline \textbf{Partecipanti:} Scrum Master (Alessia Crimaldi), Dev 1 (Alessio Arcara), Dev 2 (Matteo Sacco)
        \newline \textbf{Percezione attuale dello SG:} Negativa (Dev 1), Positiva (Dev 2)
    \end{itemize}
    \begin{itemize}
        \item \textbf{Data:} 11/01/2023
        \newline \textbf{Durata:} 15 min.
        \newline \textbf{Partecipanti:} Scrum Master (Alessia Crimaldi), Dev 1 (Alessio Arcara), Dev 2 (Matteo Sacco)
        \newline \textbf{Percezione attuale dello SG:} Negativa (Dev 1), Negativa (Dev 2)
    \end{itemize}
    \begin{itemize}
        \item \textbf{Data:} 12/01/2023
        \newline \textbf{Durata:} 15 min.
        \newline \textbf{Partecipanti:} Scrum Master (Alessia Crimaldi), Dev 1 (Alessio Arcara), Dev 2 (Matteo Sacco)
        \newline \textbf{Percezione attuale dello SG:} Negativa (Dev 1), Negativa (Dev 2)
    \end{itemize}
    \begin{itemize}
        \item \textbf{Data:} 13/01/2023
        \newline \textbf{Durata:} 12 min.
        \newline \textbf{Partecipanti:} Scrum Master (Alessia Crimaldi), Dev 1 (Alessio Arcara), Dev 2 (Matteo Sacco)
        \newline \textbf{Percezione attuale dello SG:} Negativa (Dev 1), Negativa (Dev 2)
    \end{itemize}


\end{document}
