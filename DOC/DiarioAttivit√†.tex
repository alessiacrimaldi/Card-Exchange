\documentclass{article}
\usepackage[utf8]{inputenc}

\title{Diario del progetto}
% \author{Alessia Crimaldi}
\date{January 2023}

\usepackage{blindtext}\usepackage[a4paper, total={6in, 8in}]{geometry}
\begin{document}

    \maketitle

    \section{SPRINT 01}

    \begin{itemize}
        \item Inizio sprint: \textit{08/01/2023}
        \item Fine sprint: \textit{14/01/2023}
    \end{itemize}

    \begin{itemize}
        \item \textbf{Sprint Goal:}
        \begin{indent}
            \newline Permettere ad un visitatore di effettuare tutte le funzionalita' base dell'app, ossia di avere una corretta ed efficace importazione e gestione del catalogo carte con le relative ricerche e dettagli.
        \end{indent}
    \end{itemize}

    \begin{itemize}
        \item \textbf{RUOLI.}
        \newline \textbf{Product Owner:} Davide Fermi
        \newline \textbf{Scrum Master:} Alessia Crimaldi
        \newline \textbf{Development team:} Alessio Arcara, Matteo Sacco
    \end{itemize}

    \subsection{Sprint Planning}
    \item \textbf{Data:} 08/01/2023
    \newline \textbf{Durata:} 95 min.
    \newline \textbf{Partecipanti:} Scrum Master (Alessia Crimaldi), Product Owner (Davide Fermi) Dev 1 (Alessio Arcara), Dev 2 (Matteo Sacco)
    \newline \textbf{Note:} Spiegazione del Product Backlog da parte del PO (user stories e relative priorità). Poker Planning tra i membri del team utilizzando sequenza di Fibonacci modificata (0  1/2  1  2  3  5  8  13  20  40  100  ?  -tazzina di caffè-). Stima dell'effort delle user stories da parte di tutto il team. opolamento dello Sprint Backlog (tasks) a cura degli sviluppatori e dello SM. Stima dell'effort dei singoli task relativi ad ogni user story da parte di tutto il team.

    \newpage

    \subsection{BOARD}
    \begin{itemize}
        \newline
        \textbf{Stato iniziale:}
        \newline
        \newline
        \begin{tabular}{ | p{2.5cm} | p{4cm} | p{2.6cm} | p{1.6cm} | p{1.6cm} | }
            \hline
            \textbf{UC raffinati}
            & \textbf{Product Backlog}
            & \textbf{Sprint Backlog}
            & \textbf{Sprint In progress}
            & \textbf{Sprint Done} \\
            \hline
            & \textbf{UC01:} Come visitatore voglio selezionare un gioco e vedere l'elenco di tutte le sue carte & & & \\
            \hline
            & \textbf{UC02:} Come visitatore voglio poter fare delle ricerche per vedere un elenco di carte filtrato & & & \\
            \hline
            & \textbf{UC03:} Come visitatore voglio selezionare una carta e vederne i dettagli & & & \\
            \hline
        \end{tabular}
        \newpage
        \textbf{Stato dopo lo Sprint Planning:}
        \newline
        \newline
        \begin{tabular}{ | p{3.5cm} | p{1.5cm} | p{5cm} | p{1.6cm} | p{1.6cm} | }
            \hline
            \textbf{UC raffinati}
            & \textbf{Product Backlog}
            & \textbf{Sprint Backlog}
            & \textbf{Sprint In progress}
            & \textbf{Sprint Done} \\
            \hline
            \textbf{UC01:} Come visitatore voglio selezionare un gioco e vedere l'elenco di tutte le sue carte
            & & \textbf{Task1 [UC01]\#2:} Creare il modello per le carte di tutti i giochi & & \\
            \hline
            \textbf{UC02:} Come visitatore voglio poter fare delle ricerche per vedere un elenco di carte filtrato
            & & \textbf{Task2 [UC01]\#3:} Importare dai JSON le carte dei giochi & & \\
            \hline
            \textbf{UC03:} Come visitatore voglio selezionare una carta e vederne i dettagli
            & & \textbf{Task3 [UC01]\#4:} Inizializzare il DB con i JSON caricati & & \\
            \hline
            & & \textbf{Task4 [UC01]\#5:} Creare RemoteService per fetch delle carte di quel gioco & & \\
            \hline
            & & \textbf{Task5 [UC01]\#6:} Creare HomeView per la visualizzazione delle carte & & \\
            \hline
            & & \textbf{Task6 [UC01]\#7:} Creare Composite Card per la visualizzazione carta & & \\
            \hline
            & & \textbf{Task1 [UC02]\#10:} Creare filtri specifici di quel gioco & & \\
            \hline
            & & \textbf{Task2 [UC02]\#11:} Filtrare l'array delle carte in base ai filtri specificati & & \\
            \hline
            & & \textbf{Task3 [UC02]\#12:} Rimuovere dalla pagina HomeView le carte che non corrispondono ai filtri & & \\
            \hline
            & & \textbf{Task1 [UC03]\#14:} Creazione CardPlace & & \\
            \hline
            & & \textbf{Task2 [UC03]\#15:} Connessione RemoteService FetchCard nel caso l'assunzione sia errata & & \\
            \hline
            & & \textbf{Task3 [UC03]\#16:} Creazione CardView e popolamento delle informazioni & & \\
            \hline
            & & \textbf{Task4 [UC03]\#17:} Gestione del cambio di pagina da HomeView a CardView & & \\
            \hline
        \end{tabular}
        \newpage
        \textbf{Stato finale:}
        \newline
        \newline
        \begin{tabular}{ | p{2.4cm} | p{1.4cm} | p{2.7cm} | p{3.4cm} | p{3cm} | }
            \hline
            \textbf{UC raffinati}
            & \textbf{Product Backlog}
            & \textbf{Sprint Backlog}
            & \textbf{Sprint In progress}
            & \textbf{Sprint Done} \\
            \hline
            \textbf{UC01:} Come visitatore voglio selezionare un gioco e vedere l'elenco di tutte le sue carte
            & & \textbf{Task1 [UC03]\#14:} Creazione CardPlace
            & \textbf{Task2 [UC02]\#11:} Filtrare l'array delle carte in base ai filtri specificati
            & \textbf{Task1 [UC01]\#2:} Creare il modello per le carte di tutti i giochi \\
            \hline
            \textbf{UC02:} Come visitatore voglio poter fare delle ricerche per vedere un elenco di carte filtrato
            & & \textbf{Task2 [UC03]\#15:} Connessione RemoteService FetchCard nel caso l'assunzione sia errata
            & \textbf{Task3 [UC02]\#12:} Rimuovere dalla pagina HomeView le carte che non corrispondono ai filtri
            & \textbf{Task4 [UC01]\#5:} Creare RemoteService per fetch delle carte di quel gioco \\
            \hline
            \textbf{UC03:} Come visitatore voglio selezionare una carta e vederne i dettagli
            & & \textbf{Task3 [UC03]\#16:} Creazione CardView e popolamento delle informazioni
            & \textbf{Task6 [UC01]\#7:} Creare Composite Card per la visualizzazione carta
            & \textbf{Task5 [UC01]\#6:} Creare HomeView per la visualizzazione delle carte \\
            \hline
            & & \textbf{Task4 [UC03]\#17:} Gestione del cambio di pagina da HomeView a CardView
            & & \textbf{Task2 [UC01]\#3:} Importare dai JSON le carte dei giochi \\
            \hline
            & & & & \textbf{Task1 [UC02]\#10:} Creare filtri specifici di quel gioco  \\
            \hline
            & & & & \textbf{Task3 [UC01]\#4:} Inizializzare il DB con i JSON caricati \\
            \hline
        \end{tabular}
    \end{itemize}

    \subsection{Daily Scrum}
    \begin{itemize}
        \item \textbf{Data:} 09/01/2023
        \newline \textbf{Durata:} 15 min.
        \newline \textbf{Partecipanti:} Scrum Master (Alessia Crimaldi), Dev 1 (Alessio Arcara)
        \newline \textbf{Percezione attuale dello SG:} Positiva (Dev 1)
    \end{itemize}
    \begin{itemize}
        \item \textbf{Data:} 10/01/2023
        \newline \textbf{Durata:} 15 min.
        \newline \textbf{Partecipanti:} Scrum Master (Alessia Crimaldi), Dev 1 (Alessio Arcara), Dev 2 (Matteo Sacco)
        \newline \textbf{Percezione attuale dello SG:} Negativa (Dev 1), Positiva (Dev 2)
    \end{itemize}
    \begin{itemize}
        \item \textbf{Data:} 11/01/2023
        \newline \textbf{Durata:} 15 min.
        \newline \textbf{Partecipanti:} Scrum Master (Alessia Crimaldi), Dev 1 (Alessio Arcara), Dev 2 (Matteo Sacco)
        \newline \textbf{Percezione attuale dello SG:} Negativa (Dev 1), Negativa (Dev 2)
    \end{itemize}
    \begin{itemize}
        \item \textbf{Data:} 12/01/2023
        \newline \textbf{Durata:} 15 min.
        \newline \textbf{Partecipanti:} Scrum Master (Alessia Crimaldi), Dev 1 (Alessio Arcara), Dev 2 (Matteo Sacco)
        \newline \textbf{Percezione attuale dello SG:} Negativa (Dev 1), Negativa (Dev 2)
    \end{itemize}
    \begin{itemize}
        \item \textbf{Data:} 13/01/2023
        \newline \textbf{Durata:} 12 min.
        \newline \textbf{Partecipanti:} Scrum Master (Alessia Crimaldi), Dev 1 (Alessio Arcara), Dev 2 (Matteo Sacco)
        \newline \textbf{Percezione attuale dello SG:} Negativa (Dev 1), Negativa (Dev 2)
    \end{itemize}

    \subsection{Sprint Review}
    \begin{itemize}
        \item \textbf{Data:} 15/01/2023
        \newline \textbf{Durata:} 65 min.
        \newline \textbf{Partecipanti:} TODO Scrum Master (Davide Fermi), Product Owner (Matteo Sacco) Dev 1 (Alessia Crimaldi), Dev 2 (Alessio Arcara)
        \newline \textbf{Note:} TODO Viene prima effettuata la demo da parte dei Devs. Nella discussione sull'andamento dello Sprint, emerge come sia necessario migliorare la definizione dei tasks, aumentandone il dettaglio e quindi suddividendo task grandi in molti sub-tasks. Questo anche per evitare Pull request troppo grosse (e più complicate da gestire). Si evidenzia che il Product Backlog andrebbe continuamente raffinato durante sprint. Ci sono critiche e chiarimenti rigurardo effort di alcuni elementi del team (in particolare verso Matteo e Davide). Viene chiesta in generale anche più precisione durante Sprint Planning, in quanto anche questa settimana si stava rischiando di non rilasciare. Viene da alcuni membri sottolineato come il livello di dettaglio dell'attuale manuale utente sia troppo profondo, richiedendo sforzi per taluni eccessivi rispetto a quanto preventivato. Il PO al termine della riunione, decide che si provvede a rilasciare e incarica i Devs delle pratiche formali di rilascio.
    \end{itemize}

    \subsection{Sprint Retrospective}
    \begin{itemize}
        \item \textbf{Data:} 15/01/2023
        \newline \textbf{Durata:} 70 min.
        \newline \textbf{Partecipanti:}  TODO Scrum Master (Davide Fermi), Product Owner (Matteo Sacco) Dev 1 (Alessia Crimaldi), Dev 2 (Alessio Arcara)
        \newline \textbf{Note:} TODO Viene proposta un nuovo ordine e dettaglio della D.O.D. (tutti si concorda e viene subito attuato nelle actions di GitHub), viene proposto di effettuare "pair programming" per agevolare i nuovi sviluppatori. Viene confermata l'importanza del TDD nonostante l'ingente quantità di effort che richiede e si decide concordi di effettuare i prossimi Sprint Planning con il metodo "velocity" per cercare di organizzare al meglio lo Sprint successivo.
    \end{itemize}



    \newpage

    \section{SPRINT 02}

    \begin{itemize}
        \item Inizio sprint: \textit{15/01/2023}
        \item Fine sprint: \textit{22/01/2023}
    \end{itemize}

    \begin{itemize}
        \item \textbf{Sprint Goal:}
        \begin{indent}
            \newline Permettere ad un visitatore di autenticarsi come utente e permettere ad un visitatore di effettuare una ricerca di carte dal catalogo potendo applicanre dei filtri
        \end{indent}
    \end{itemize}

    \begin{itemize}
        \item \textbf{RUOLI.}
        \newline \textbf{Product Owner:} Matteo Sacco
        \newline \textbf{Scrum Master:} Davide Fermi
        \newline \textbf{Development team:} Alessio Arcara, Alessia Crimaldi
    \end{itemize}

    \subsection{Sprint Planning}
    \begin{itemize}
        \item \textbf{Data:} 15/01/2023
        \newline \textbf{Durata:} 80 min.
        \newline \textbf{Partecipanti:} Scrum Master (Davide Fermi), Product Owner (Matteo Sacco) Dev 1 (Alessia Crimaldi), Dev 2 (Alessio Arcara)
        \newline \textbf{Note:} Effettuata presentazione Product Backlog da parte del PO, PokerPlain per definire Story Points, Stima effort per singuli UC, previsione effort totale settimana, Inserimento UC in Sprint Backlog a seconda di stima UC / effort totale disponibile, raffinamento UC in tasks, stima singoli tasks.
    \end{itemize}


    \begin{itemize}
        \newpage
        \subsection{BOARD}
        \textbf{Stato dopo lo Sprint Planning:}
        \newline
        \newline
        \begin{tabular}{ | p{3.5cm} | p{1.5cm} | p{5cm} | p{1.6cm} | p{1.6cm} | }
            \hline
            \textbf{UC raffinati}
            & \textbf{Product Backlog}
            & \textbf{Sprint Backlog}
            & \textbf{Sprint In progress}
            & \textbf{Sprint Done} \\
            \hline
            \textbf{UC03:}  Come visitatore voglio selezionare una carta e vederne i dettagli
            & & \textbf{Task1 [UC03]\#14:} Creazione CardPlace & & \\
            \hline
            \textbf{UC04:}  Come utente voglio potermi autenticare per gestire i miei mazzi di carte
            & & \textbf{Task2 [UC03]\#15:} Connessione RemoteService FetchCard nel caso l'assunzione sia errata & & \\
            \hline
            \textbf{UC05:}  Come utente voglio poter aggiungere le mie carte reali al mazzo delle possedute
            & & \textbf{Task3 [UC03]\#16:} Creazione CardView e popolamento delle informazioni & & \\
            \hline
            & & \textbf{Task4 [UC03]\#17:} Gestione del cambio di pagina da HomeView a CardView & & \\
            \hline
            & & \textbf{Task1 [UC04]\#32:} Creare Authentication view per Login/registrazione  & & \\
            \hline
            & & \textbf{Task2 [UC04]\#33:} Creare RPC SignUp & & \\
            \hline
            & & \textbf{Task3 [UC04]\#34:} Creare RPC SignIn  & & \\
            \hline
            & & \textbf{Task4 [UC04]\#35:} Gestire la sessione  & & \\
            \hline
            & & \textbf{Task1 [UC05]\#36:} Creare modello per strutturare i mazzi predefiniti (possedute/desiderate) e il loro contenuto  & & \\
            \hline
            & & \textbf{Task2 [UC05]\#54:}Creare mappa dei mazzi su MapDB  & & \\
            \hline
            & & \textbf{Task3 [UC05]\#63:} Creare metodo statico in AuthService per controllare validità token e restituire email utente  & & \\
            \hline
            & & \textbf{Task4 [UC05]\#55:} Modificare RPC signup per aggiungere il mazzo delle possedute alla creazione dell'utente  & & \\
            \hline
            & & \textbf{Task5 [UC05]\#56:} Creare RPC per inserire una carta fisica nel mazzo delle possedute dell'utente  & & \\
            \hline
            & & \textbf{Task6 [UC05]\#57:} Modificare CardView per poter aggiungere la carta al mazzo delle possedute [UC05] #57 & & \\
            \hline
        \end{tabular}
        \newpage
        \textbf{Stato finale:}
        \newline
        \newline
        \begin{tabular}{ | p{2.4cm} | p{1.4cm} | p{2.7cm} | p{3.4cm} | p{3cm} | }
            \hline
            \textbf{UC raffinati}
            & \textbf{Product Backlog}
            & \textbf{Sprint Backlog}
            & \textbf{Sprint In progress}
            & \textbf{Sprint Done} \\
            \hline
            \textbf{UC01:} Come visitatore voglio selezionare un gioco e vedere l'elenco di tutte le sue carte
            & & \textbf{Task1 [UC03]\#14:} Creazione CardPlace
            & \textbf{Task2 [UC02]\#11:} Filtrare l'array delle carte in base ai filtri specificati
            & \textbf{Task1 [UC01]\#2:} Creare il modello per le carte di tutti i giochi \\
            \hline
            \textbf{UC02:} Come visitatore voglio poter fare delle ricerche per vedere un elenco di carte filtrato
            & & \textbf{Task2 [UC03]\#15:} Connessione RemoteService FetchCard nel caso l'assunzione sia errata
            & \textbf{Task3 [UC02]\#12:} Rimuovere dalla pagina HomeView le carte che non corrispondono ai filtri
            & \textbf{Task4 [UC01]\#5:} Creare RemoteService per fetch delle carte di quel gioco \\
            \hline
            \textbf{UC03:} Come visitatore voglio selezionare una carta e vederne i dettagli
            & & \textbf{Task3 [UC03]\#16:} Creazione CardView e popolamento delle informazioni
            & \textbf{Task6 [UC01]\#7:} Creare Composite Card per la visualizzazione carta
            & \textbf{Task5 [UC01]\#6:} Creare HomeView per la visualizzazione delle carte \\
            \hline
            & & \textbf{Task4 [UC03]\#17:} Gestione del cambio di pagina da HomeView a CardView
            & & \textbf{Task2 [UC01]\#3:} Importare dai JSON le carte dei giochi \\
            \hline
            & & & & \textbf{Task1 [UC02]\#10:} Creare filtri specifici di quel gioco  \\
            \hline
            & & & & \textbf{Task3 [UC01]\#4:} Inizializzare il DB con i JSON caricati \\
            \hline
        \end{tabular}
    \end{itemize}

    \subsection{Daily Scrum}
    \begin{itemize}
        \item \textbf{Data:} 16/01/2023
        \newline \textbf{Durata:} 20 min.
        \newline \textbf{Partecipanti:} Scrum Master (Davide Fermi), Dev 1 (Alessia Crimaldi), Dev 2 (Alessio Arcara)
        \newline \textbf{Percezione attuale dello SG:} Positiva (Dev 1), Positiva (Dev 1)
    \end{itemize}
    \begin{itemize}
        \item \textbf{Data:} 17/01/2023
        \newline \textbf{Durata:} 15 min.
        \newline \textbf{Partecipanti:} Scrum Master (Davide Fermi), Dev 1 (Alessia Crimaldi), Dev 2 (Alessio Arcara)
        \newline \textbf{Percezione attuale dello SG:} Positiva (Dev 1), Positiva (Dev 1)
    \end{itemize}
    \begin{itemize}
        \item \textbf{Data:} 18/01/2023
        \newline \textbf{Durata:} 15 min.
        \newline \textbf{Partecipanti:} Scrum Master (Davide Fermi), Dev 1 (Alessia Crimaldi), Dev 2 (Alessio Arcara)
        \newline \textbf{Percezione attuale dello SG:} Positiva (Dev 1), Positiva (Dev 1)
    \end{itemize}
    \begin{itemize}
        \item \textbf{Data:} 19/01/2023
        \newline \textbf{Durata:} 15 min.
        \newline \textbf{Partecipanti:} Scrum Master (Davide Fermi), Dev 1 (Alessia Crimaldi), Dev 2 (Alessio Arcara)
        \newline \textbf{Percezione attuale dello SG:} Negativa (Dev 1), Negativi (Dev 1)
        \newline \textbf{Note:} Viene richiesta convocazione PO per rivalutazione Sprint Backlog
    \end{itemize}
    \begin{itemize}
        \item \textbf{Data:} 20/01/2023
        \newline \textbf{Durata:} 45 min.
        \newline \textbf{Partecipanti:} Scrum Master (Davide Fermi), Product Owner (Matteo Sacco), Dev 1 (Alessia Crimaldi), Dev 2 (Alessio Arcara)
        \newline \textbf{Percezione attuale dello SG:} Positiva (Dev 1), Positiva (Dev 1)
        \newline \textbf{Note:} Viene ridiscusso sprint backlog con PO in base a problematiche emerse.
    \end{itemize}
    \begin{itemize}
        \item \textbf{Data:} 21/01/2023
        \newline \textbf{Durata:} 15 min.
        \newline \textbf{Partecipanti:} Scrum Master (Davide Fermi), Dev 1 (Alessia Crimaldi), Dev 2 (Alessio Arcara)
        \newline \textbf{Percezione attuale dello SG:} Negativa (Dev 1), Positiva (Dev 1)
    \end{itemize}

    \subsection{Sprint Review}
    \begin{itemize}
        \item \textbf{Data:} 22/01/2023
        \newline \textbf{Durata:} 65 min.
        \newline \textbf{Partecipanti:} Scrum Master (Davide Fermi), Product Owner (Matteo Sacco) Dev 1 (Alessia Crimaldi), Dev 2 (Alessio Arcara)
        \newline \textbf{Note:}Viene prima effettuata la demo da parte dei Devs. Nella discussione sull'andamento dello Sprint, emerge come sia necessario migliorare la definizione dei tasks, aumentandone il dettaglio e quindi suddividendo task grandi in molti sub-tasks. Questo anche per evitare Pull request troppo grosse (e più complicate da gestire). Si evidenzia che il Product Backlog andrebbe continuamente raffinato durante sprint. Ci sono critiche e chiarimenti rigurardo effort di alcuni elementi del team (in particolare verso Matteo e Davide). Viene chiesta in generale anche più precisione durante Sprint Planning, in quanto anche questa settimana si stava rischiando di non rilasciare. Viene da alcuni membri sottolineato come il livello di dettaglio dell'attuale manuale utente sia troppo profondo, richiedendo sforzi per taluni eccessivi rispetto a quanto preventivato. Il PO al termine della riunione, decide che si provvede a rilasciare e incarica i Devs delle pratiche formali di rilascio.
    \end{itemize}

    \subsection{Sprint Retrospective}
    \begin{itemize}
        \item \textbf{Data:} 22/01/2023
        \newline \textbf{Durata:} 70 min.
        \newline \textbf{Partecipanti:} Scrum Master (Davide Fermi), Product Owner (Matteo Sacco) Dev 1 (Alessia Crimaldi), Dev 2 (Alessio Arcara)
        \newline \textbf{Note:}Viene proposta un nuovo ordine e dettaglio della D.O.D. (tutti si concorda e viene subito attuato nelle actions di GitHub), viene proposto di effettuare "pair programming" per agevolare i nuovi sviluppatori. Viene confermata l'importanza del TDD nonostante l'ingente quantità di effort che richiede e si decide concordi di effettuare i prossimi Sprint Planning con il metodo "velocity" per cercare di organizzare al meglio lo Sprint successivo.
    \end{itemize}

\end{document}
